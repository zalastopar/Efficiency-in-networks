\documentclass[a4paper, 16pt]{article}
\usepackage[slovene]{babel}
\usepackage[utf8]{inputenc}
\usepackage[T1]{fontenc}
\usepackage{lmodern}

\title{Učinkovitost omrežij}
\date{2020\\ November}
\author{Jure Babnik \\  Zala Stopar Špringer}


\begin{document}

\maketitle

\section{Opis problema}

Učinkovitost omrežja je merilo učinkovitosti izmenjave informacij v omrežju. Poznamo lokalno in globalno učinkovitost. Globalna učinkovitost nam pove, kako dobro si celotno omrežje
izmenjuje informacije, ko se le-te izmenjujejo hkrati. Lokalna učinkovitost pa nam pove, kako odporno je omrežje, v primeru odpovedi nekega majhnega dela omrežja.\\

Formula za \textbf{povprečno učinkovitost} grafa $G$ je definirana kot:
$$ E(G) = \frac{1}{n(n-1)} \sum_{i\neq j \in G} \frac{1}{d(i,j)},$$

kjer je $d(i,j)$ dolžina najkrajše poti med $i$-to in $j$-to točko, $n$ pa je število vseh točk v grafu.\\

\textbf{Globalna učinkovitost} je definirana kot:
$$ E_{glob}(G) = \frac{E(G)}{E(K_n)}, $$

kjer $K_n$, predstavlja poln graf na $n$ točkah.


\textbf{Lokalna učinkovitost} je definirana kot:
$$ E_{loc}(G) = \frac{1}{n} \sum_{i \in G} E(G_i), $$

kjer $G_i$ predstavlja podgraf grafa $G$, ki je sestavljen le iz sosedov točke $i$ (brez točke $i$).

Izračunala bova učinkovitost omrežij preprostih grafov, kot so mreže dimenzije $1 x n$, $2 x n$, $n x n$, 3-dimenzionalne mreže, cikli, binomska drevesa, itd.
Za preprostejše grafe bova učinkovitost izračunala točno, rezultat pa primerjala s približkom, ki ga dobimo tako, da uporabimo le razdalje med naključnimi pari točk, 
rezultat pa posplošimo na celoten graf. Pogledala si bova tudi, kako se povprečna učinkovitost spremeni, če odstranimo katero od točk. 

\section{Programsko okolje in implementacija}

Izbrala sva programski jezik \emph{Python}, saj nama je obema najbolj domač. Uporabiti nameravava knjižnici \texttt{graph-theory} in (po priporočilu asistenta) \texttt{networkx}.
V knjižnicah so že definirani razredi, ki jih bova potrebovala za raziskovanje in prikaz grafov. 

\section{Načrt za delo}

Prvi korak je ustvarjanje grafov. Preprostejše grafe nameravava zgraditi vsakega posebej, nato pa nameravava napisati algoritem za strukturiranje naključnega grafa z $n$ vozlišči.
Po potrebi bova algoritmu dodajala podrobnejša navodila (npr. naj obstaja pot med poljubnima dvema točkama). Računanje razdalje med točkama ne bo velik problem, saj knjižnice
že vsebujejo funkcije oz. metode, namenjene prav temu. Z uporabo teh funkcij lahko napiševa funkcijo, ki bo računala povprečno učinkovitost, z uporabo te funkcije pa lahko potem
izračunava tudi globalno in lokalno učinkovitost. Potrebovala bova tudi funkcijo, ki bo naključno izbrala nekaj vozlišč grafa in jih odstranila (skupaj z vsemi njihovimi povezavami).
Nazadnje bova napisala še fukncijo, ki računa učinkovitost omrežja na podlagi naključnega podgrafa, rezultat pa posplošila na celotno omrežje (s tem bova raziskala smiselnost
tovrstnega pristopa računanja učinkovitosti omrežja).



\end{document}