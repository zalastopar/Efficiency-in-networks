\documentclass[a4paper, 16pt]{article}
\usepackage[slovene]{babel}
\usepackage[utf8]{inputenc}
\usepackage[T1]{fontenc}
\usepackage{lmodern}

\title{Učinkovitost v omrežjih}
\date{2020\\ November}
\author{Jure Babnik \\  Zala Stopar Špringer}


\begin{document}

\maketitle

\section{Opis problema}

Učinkovitost omrežja je merilo učinkovitosti izmenjave informacij v omrežju. Poznamo lokalno in globalno učinkovitost. Globalna učinkovitost nam pove, kako dobro si celotno omrežje
izmenjuje informacije, ko se le-te izmenjujejo hkrati. Lokalna učinkovitost pa nam pove, kako odporno je omrežje, v primeru odpovedi nekega majhnega dela omrežja.\n

Formula za \textbf{povprečno učinkovitost} grafa $G$ je definirana kot:
$$ E(G) = \frac{1}{n(n-1)} \sum_{i\neq j \in G} \frac{1}{d(i,j)}$$,

kjer je $d(i,j)$ dolžina najkrajše poti med $i$-to in $j$-to točko, $n$ pa je število vseh točk v grafu.\n
\textbf{Globalna učinkovitost} je definirana kot:
$$ E_{glob}(G) = \frac{E(G)}{E(K_n)} $$,

kjer $K_n$, predstavlja poln graf na $n$ točkah.

\textbf{Lokalna učinkovitost} je definirana kot:
$$ E_{loc}(G) = \frac{1}{n} \sum_{i \in G} E(G_i) $$,

kjer $G_i$ predstavlja podgraf grafa $G$, ki je sestavljen le iz sosedov točke $i$ (brez točke $i$).\n

Izračunala bova učinkovitost omrežij preprostih grafov, kot so mreže dimenzije $1 x n$, $2 x n$, $n x n$, 3-dimenzionalne mreže, cikli, binomska drevesa, itd.
Za preprostejše grafe bova učinkovitost izračunala točno, rezultat pa primerjala s približkom, ki ga dobimo tako, da uporabimo le razdalje med naključnimi pari točk, 
rezultat pa posplošimo na celoten graf. Pogledala si bova tudi, kako se povprečna učinkovitost spremeni, če odstranimo katero od točk. 




\end{document}

